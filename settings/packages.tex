\usepackage{tikz-cd}
\usepackage[ruled,vlined]{algorithm2e}

% настройка кодировки, шрифтов и русского языка
\usepackage{fontspec}
\usepackage{tipa}
\usepackage{polyglossia}
\usepackage{wasysym}
% рабочие ссылки в документе
\usepackage{hyperref}

% графика
\usepackage{graphicx}

% качественные листинги кода
\usepackage{minted}
\usepackage{listings}
\usepackage{lstfiracode}

% отключение копирования номеров строк из листинга, работает не во всех просмотрщиках (в Adobe Reader работает)
\usepackage{accsupp}
\newcommand\emptyaccsupp[1]{\BeginAccSupp{ActualText={}}#1\EndAccSupp{}}
\let\theHFancyVerbLine\theFancyVerbLine
\def\theFancyVerbLine{\rmfamily\tiny\emptyaccsupp{\arabic{FancyVerbLine}}}

 
\usepackage{apacite}
\usepackage{natbib}
\renewcommand{\bibsection}{\subsubsection*{\bibname } }   
\usepackage{caption} 
\usepackage{subcaption}
\usepackage{adjustbox}
\usepackage{float}

% разное для математики
\usepackage{amsmath, amsfonts, amssymb, amsthm, mathtools, stmaryrd}
\usepackage{gb4e}
\usepackage{textgreek}


% водяной знак на документе, см. main.tex
\usepackage[printwatermark]{xwatermark} 

% для презентаций
% \usepackage{here}
% \usepackage{animate}
% \usepackage{bm}

\usepackage{booktabs}
% \usepackage[all]{xy}


\usepackage{tikz}

\usepackage{pgf}

\usetikzlibrary{chains,fit,shapes,shapes.symbols,shapes.geometric,scopes,quotes}
\usetikzlibrary{automata,arrows,matrix,backgrounds,calc,positioning}
\tikzset{->, 
>=stealth, % makes the arrow heads bold
% node distance=3cm, % specifies the minimum distance between two nodes. Change if necessary.
every state/.style={}, % sets the properties for each ’state’ node
every loop/.style={min distance=9mm,in=80,out=100,looseness=6},
initial text=$ $, % sets the text that appears on the start arrow
}

\usetikzlibrary{decorations.pathreplacing, arrows.meta}


\usepackage{xspace}

\usepackage[ruled,vlined]{algorithm2e}
